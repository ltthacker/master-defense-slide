%!TEX root = ../main.tex
\section{Thuật toán đề xuất}

\begin{frame}{Cơ chế chọn tác vụ hỗ trợ - MAB}
    \begin{alertblock}{Mô hình hóa việc chọn tác vụ để lai ghép bằng MAB}
        \begin{mydef}[Lựa chọn]
            Với mỗi tác vụ $T_k$, sẽ có $K-1$ lựa chọn, tương ứng với $K-1$ tác vụ $T_{k'}$ mà $k' \in \{1, \ldots, K\} \text{ và } k' \ne k$.
            \label{def:propose:action}
        \end{mydef}

        \begin{mydef}[Phần thưởng]
            Sau khi tác vụ $T_{k'}$ được lựa chọn để ghép cặp với tác vụ $T_{k}$, phần thưởng của việc chọn tác vụ $T_{k'}$ được định nghĩa như sau:
            \begin{equation}
                r(k, k') = \left\{
                    \begin{array}{ll}
                        1 \text{ nếu } f_k(c) < f_k(p) , \exists p \in P^k \\\
                        0 \text{ trong các trường hợp khác}.
                    \end{array}
                  \right.
            \end{equation}
            \begin{itemize}
                \item $c$ là con sinh ra trong quá trình lai ghép khác tác vụ
                \item $f_k(.)$ là hàm đánh giá của tác vụ $T_k$
            \end{itemize}
            \label{def:propose:reward}
        \end{mydef}
    \end{alertblock}
\end{frame}

\begin{frame}{Cách giải bài toán con chọn tác vụ hỗ trợ - KLUCB}
    \begin{block}{Giả định}
        \begin{itemize}
            \item \textbf{Phần thưởng}: Biến ngẫu nhiên với giá trị $\{0, 1\}$
            \item \textbf{Giả định}: Phần thưởng sinh từ phân phối Bernoulli chưa biết trước.
        \end{itemize}
    \end{block}
    \begin{block}{Cách giải - KLUCB}
        \begin{equation}
            k' = \underset{j}{\text{argmax }} \mu(j) + \frac{1 + t \times log^2(t) }{N(j)}
            \label{eq:propose:klucb}
        \end{equation}
        \begin{itemize}
            \item $\mu(j)$ là giá trị trung bình ước lượng được của phần thưởng khi lựa chọn $j$
            \item $N(j)$ là tổng số lần thuật toán đã lựa chọn $j$
            \item $t$ là tổng số của tất cả các lần lựa chọn
        \end{itemize}
    \end{block}
    \begin{block}{Tham khảo}
        \fullcite{lattimore2020bandit}
    \end{block}
\end{frame}

\begin{frame}{Cấu trúc cập nhật tuần tự}
    Viết cái giải thuật mô tả
\end{frame}

\begin{frame}{Cấu trúc cập nhật tuần tự}
    Vẽ cái nguyên lý, giải thích lý do.
\end{frame}

\begin{frame}{Tóm tắt}
    Tóm tắt những ý tốt xấu so với các nghiên cứu liên quan
\end{frame}

\begin{frame}{Áp dụng - Tối ưu nhiều mạng nơ-ron}
    Tổng kết vào đây.
\end{frame}

% \begin{frame}{Task selection as a Multi Armed Bandit (MAB) problem}
%     \begin{block}{Given}
%         \begin{itemize}
%             \item A $K$-task MTO problem
%         \end{itemize}
%     \end{block}
%     \begin{mydef}[Action]
%         For a task $k$, there are $K-1$ actions of choosing $k'$ such that $k' \in \{1, \ldots, K\} \text{ and } k' \ne k$.
%         \label{def:action}
%     \end{mydef}
%     \begin{mydef}[Reward]
%         After task $k'$ is selected to be combined with task $k$, the reward of choosing that action is defined as 
%         \begin{equation}
%             reward = \left\{
%                 \begin{array}{ll}
%                     1 \text{ if } f_k(c) < f_k(p) , \exists p \in P_k \\\
%                     0 \text{ otherwise}.
%                 \end{array}
%               \right.
%         \end{equation}
%         where $c$ is the offspring generated by the reproduction procedure and $f_k(.)$ is the fitness function of the $k^{th}$ task.
%         \label{def:reward}
%     \end{mydef}
% \end{frame}

% \begin{frame}{UCB function to solve MAB}
%     \begin{block}{Property of reward function}
%         \begin{itemize}
%             \item Reward takes two value $0$ or $1$ $\rightarrow$ reward distribution is generated from an unknown Bernoulli distribution.
%             \item From \footfullcite{lattimore2020bandit}, use KL-UCB to solve.
%         \end{itemize}
%     \end{block}
%     \begin{block}{KL-UCB function}
%         \begin{equation}
%             k' = \underset{j}{\text{argmin }} \mu(j) + \frac{1 + t \times log^2(t) }{N(j)}
%             \label{eq:klucb}
%         \end{equation}
%         where 
%         \begin{itemize}
%             \item $\mu(j)$ is the mean of reward when task $j$ is chosen
%             \item $N(j)$ is the number of times task $j$ is chosen
%             \item $t$ is the total number of actions chosen.
%         \end{itemize}
%     \end{block}
% \end{frame}

% \begin{frame}{\gls{propose}, optimize 1 task, 1 generation}
%     \begin{algorithm}[H]
%         \fontsize{6pt}{10}\selectfont
%         \caption{\fontsize{6pt}{10}\selectfont\gls{propose} on each generation of $k^{th}$ task}
%         \begin{algorithmic}[1]
%             \State Initialize $P^{(c)}_k=\emptyset$;
%             \While{number of offspring $< N$}
%                 \State Randomly select $p_a$ from $P_k$;
%                 \If{$rand(0, 1) < rmp$}
%                     \State Choose $k'$ using Formula \eqref{eq:klucb};
%                     \State Randomly select $p_b$ from $P_{k'}$;
%                     \State $c = $ \emph{Inter-task crossover} between  $p_a$ and $p_b$;
%                 \Else
%                     \State Sample $p_b$ from $P_k$;
%                     \State $c = $ \emph{Intra-task crossover} between  $p_a$ and $p_b$;
%                 \EndIf
%                 \State $c = mutate(c)$;
%                 \State Evaluate offspring $c$;
%                 \State Update estimation $\mu(k')$ and $N(k')$ for \gls{klucb} if $c$ generated by inter-task crossover;
%                 \State $P_k^{(c)} = P_c \cup \{c\}$;
%             \EndWhile
%             \State $P_k \leftarrow$ Select $N$ best individuals from $P_k \cup P_k^{(c)}$ to form the next-population of task $k$;
%         \end{algorithmic}
%     \end{algorithm}
% \end{frame}


% \begin{frame}{\gls{propose}, proposed structure for manytasking}
%     \begin{algorithm}[H]
%         \caption{Pseudo code of \gls{propose}}
%         \begin{algorithmic}[1]
%             \For{$k \in \{1, \ldots, K\}$}
%                 \State Randomly sample $N$ individuals to form subpopulation $P_k$;
%                 \State Evaluate individual in $P_k$ for task $k$ only;
%             \EndFor
%             \While{stopping conditions are not satisfied}
%                 \For{$k \in random\_permutation({1, \ldots, K})$}
%                     \State Invoke Algorithm 1 for task $k$;
%                 \EndFor
%             \EndWhile
%         \end{algorithmic}
%     \end{algorithm}
% \end{frame}
